\chapter*{Abstract}

In this thesis we are going to elaborate on G\"odel's Incompleteneness Theorems. G\"odel's First Incompleteness Theorem states that every theory with enough arithmetic is incomplete. This implies that there does not exist a deductive system which proves all of the true sentences in the arithmetic $\mathcal{N}=(\mathbb{N};+,\cdot, S; 0)$. G\"odel's Second Incompleteness Theorem depicts that any theory with enough arithmetic which is consistent cannot prove its own consistency.

In the first part of this thesis we prove G\"odels First Incompleteness Theorem. Hereby, we construct a sentence in a theory with enough arithmetic $\sigma$ which says about itself that it is unprovable in this theory. In the second part we utilize the Formalized First Theorem in order to show the Second Incompleteness Theorem. Lastly, we examine a similar incompleteness result from algorithmic complexity involving the Kolmogorov Complexity. The Kolmogorov Complexity of a string can be defined as the minimum length of a program that outputs that string and stops. Chaitin's Incompleteness Theorem states that there exists a number $c$ such that any consistent theory with enough arithmetic cannot prove that a string has Kolmogorov complexity larger than $c$. 

\newpage
\newpage

\begin{german}
{\let\clearpage\relax\chapter*{Zusammenfassung}}
In dieser Bachelorarbeit werden wir die G\"odelschen Unvollst\"andigkeitss\"atze behandeln. G\"odels Erster Unvollst\"andigkeitssatz besagt, dass jede Theorie mit genügend Arithmetik unvollst\"andig ist. Dies impliziert, dass es keinen Kalk\"ul gibt, in dem alle in $\mathcal{N}=(\mathbb{N};+,\cdot, S; 0)$ geltenden Aussagen beweisbar sind. G\"odels Zweiter Unvollst\"andigkeitssatz zeigt, dass jede konsistente Theorie mit genügend Arithmetik nicht beweisen kann, dass sie konsistent ist.

Im ersten Teil der Arbeit beweisen wir G\"odels Ersten G\"odelschen Unvollst\"andigkeitssatz. In einer Theorie mit gen\"ugend Arithmetik konstruieren wir einen Satz $\sigma$, welcher \"uber sich selbst sagt, dass er nicht beweisbar in dieser Theorie ist. Im zweiten Teil verwenden wir den Formalen Ersten Satz um den Zweiten Unvollst\"andigkeitssatz zu beweisen. Zuletzt schauen wir uns ein \"ahnliches Unvollst\"andigkeitsresultat aus der algorithmischen Komplexit\"at an, welches die Kolmogorov Komplexit\"at involviert. Die Kolmogorov Komplexit\"at von einem Wort ist definiert als die minimale L\"ange eines Programms, welches dieses Wort ausgibt und dabei stoppt. Chaitins Unvollst\"andigkeitssatz besagt, dass eine Zahl $c$ existiert, sodass eine beliebige konsistente Theorie mit gen\"ugend Arithmetik nicht beweisen kann, dass ein Wort eine h\"ohere Kolmogorov Komplexit\"at hat als $c$. 
\end{german}